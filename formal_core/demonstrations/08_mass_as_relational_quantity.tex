\section{Mass as a Relational Quantity}

\subsection{Problem Class}

Mass is traditionally treated as an intrinsic property of objects.
Within the MACHWERK framework, mass is not intrinsic.

Mass is a relational projection quantity.

The framework does not redefine mass numerically,
but constrains how and where mass-related statements are admissible.

\subsection{Absence of Mass in $\mathcal{U}$}

The relational full space $\mathcal{U}$ contains:

\begin{itemize}
\item no absolute mass values
\item no object-inherent inertia
\item no privileged mass scale
\end{itemize}

Only relations between processes and interaction rates exist.

\subsection{Emergence via Projection}

Let
\[
\Pi_m : \mathcal{U} \rightarrow m_2
\]
be a projection into an observable calculation space
where stable dynamical comparisons are possible.

Mass emerges as a proportionality factor
between relational process rates:
\[
m \sim \frac{R_{\text{interaction}}}{R_{\text{response}}}
\]

Mass therefore encodes resistance to relational change,
not substance.

\subsection{Relational Interpretation}

Equal mass values do not imply identical relational states.

Distinct configurations
\[
u_1 \neq u_2 \in \mathcal{U}
\]
may satisfy
\[
\Pi_m(u_1) = \Pi_m(u_2)
\]

Observed mass equality is a projection coincidence,
not an ontological statement.

\subsection{Context Dependence}

Mass values are valid only within a fixed projection domain.

Changes in relational embedding
may alter effective mass without altering formal consistency.

This allows a unified formal interpretation of:

\begin{itemize}
\item inertial mass
\item gravitational mass
\item effective mass in condensed systems
\end{itemize}

without introducing separate ontologies.

\subsection{Black Boundary for Mass}

Define
\[
\Sigma_m := \{ u \in \mathcal{U} \mid \Pi_m \text{ is non-injective} \}
\]

Beyond $\Sigma_m$:

\begin{itemize}
\item mass values remain computable
\item dynamics remain formally expressible
\item unique attribution of inertia is lost
\end{itemize}

Mass persists numerically
but loses unambiguous physical interpretation.

\subsection{CRA Constraint}

A mass-related statement $S(m)$ is admissible
only if it depends exclusively on projected relational ratios
and remains invariant under relational substitutions
that preserve $\Pi_m$.

Statements attributing absolute or intrinsic mass
are physically inadmissible under CRA.

\subsection{Formal Conclusion}

Mass in MACHWERK is not a substance property.

It is a projection-dependent relational coefficient
encoding resistance within observable dynamics.

Beyond the black boundary,
mass remains calculable but ceases to uniquely identify
a relational configuration.
