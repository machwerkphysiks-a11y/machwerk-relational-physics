\section{Gravitational Boundary Test}

This test examines gravitational statements
at the loss of projective injectivity.

\subsection{Setup}

Let
\[
\Pi : \mathcal{U} \rightarrow m_2
\]
be a projection into an observable calculation space.

Assume a relational configuration
\[
u \in \mathcal{U}
\]
encoding mass-attribution relations.

\subsection{Injective Regime}

If $\Pi$ is injective,
gravitational mass assignments are uniquely reconstructible.
Standard gravitational dynamics remain admissible.

\subsection{Boundary Regime}

At the black boundary $\Sigma$,
distinct relational configurations
\[
u_1 \neq u_2
\]
satisfy
\[
\Pi(u_1) = \Pi(u_2)
\]

Gravitational observables remain defined,
but their relational origin is ambiguous.

\subsection{Admissibility}

Gravitational force laws remain formally valid,
but any statement about underlying mass distribution
is physically inadmissible.

\subsection{Conclusion}

Gravitation persists as an observable effect,
while ontological mass attribution collapses
at the projection boundary.
