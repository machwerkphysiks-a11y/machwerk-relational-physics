\section{Time Correlation Test}

This test examines time-correlated statements
when time is treated as a derived relational quantity.

\subsection{Setup}

Let
\[
\Pi : \mathcal{U} \rightarrow m_2
\]
be a projection into an observable domain.

Assume time parameters $t_1, t_2$
defined via relational process comparisons.

\subsection{Derived Nature of Time}

Time is not a primitive element of $\mathcal{U}$.

It arises through stable relational ordering
within injective projection domains.

\subsection{Boundary Condition}

Beyond the black boundary $\Sigma$,
ordering relations may persist formally,
but lose unique reconstruction.

Statements asserting absolute temporal ordering
between non-injectively projected states
violate CRA admissibility.

\subsection{Admissibility Criterion}

A time-correlated statement $T(u)$ is admissible only if:
\[
\Pi(u_1) = \Pi(u_2)
\;\Rightarrow\;
T(u_1) = T(u_2).
\]

\subsection{Result}

Temporal correlations remain meaningful
only within injective projection domains.

Formal extensions beyond these domains
must not be physically interpreted.

\subsection{Conclusion}

Time is a relational construct whose physical meaning
is bounded by projection stability.
