\section{Gravitation as Projection Constraint}

This application treats gravitation
as a constraint emerging from projection structure,
not as a fundamental interaction.

\subsection{Setup}

Let
\[
\Pi : \mathcal{U} \rightarrow m_2
\]
be a projection into an observable spacetime domain.

Assume a relational configuration
\[
u \in \mathcal{U}
\]
with no intrinsic notion of distance or mass.

\subsection{Emergence of Mass}

Mass is not defined in $\mathcal{U}$.
It appears only as a stable parameter
in projected relational ratios.

Observed mass corresponds to
persistent relational constraints
under repeated projection.

\subsection{Constraint Interpretation}

Gravitational effects arise when
projected relational states
are forced to remain compatible
across multiple reference processes.

This produces an effective attraction
without requiring a force term in $\mathcal{U}$.

\subsection{Projection Stability}

Regions with high relational constraint density
produce stable projection artifacts
interpreted as massive structures.

The apparent curvature of trajectories
reflects projection consistency,
not geometric deformation of $\mathcal{U}$.

\subsection{Black Boundary Aspect}

Near regions of extreme constraint density,
projection injectivity degrades.

Multiple relational configurations
map to indistinguishable observable states.

This marks the approach to the black boundary.

\subsection{Admissibility}

Gravitational descriptions remain admissible
as long as projected relations
remain uniquely reconstructible.

Beyond this point,
formal continuation is possible,
but physical interpretation is suspended.

\subsection{Result}

Gravitation is modeled as a projection-induced
consistency constraint,
not as a fundamental interaction
acting within the relational domain.
