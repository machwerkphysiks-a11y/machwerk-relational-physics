\section{Projection Admissibility}

This application demonstrates the minimal conditions under which
a formal relational statement becomes physically admissible.

\subsection{Setup}

Let
\[
\mathcal{U}
\]
denote the relational reference space.

Let
\[
\Pi : \mathcal{U} \rightarrow m_2
\]
be a projection into an observable calculation domain.

Consider a relational expression
\[
S(u)
\quad\text{with}\quad
u \in \mathcal{U}.
\]

\subsection{Admissibility Condition}

A statement is physically admissible if and only if
its value depends exclusively on the projected state.

Formally, admissibility requires:
\[
\forall u_1,u_2 \in \mathcal{U} :
\Pi(u_1) = \Pi(u_2)
\;\Rightarrow\;
S(u_1) = S(u_2).
\]

\subsection{Interpretation}

This condition ensures that the statement does not distinguish
between relational states that are observationally identical.

Any dependence on unobservable relational degrees of freedom
renders the statement physically inadmissible.

\subsection{Boundary Case}

If there exist
\[
u_1 \neq u_2
\quad\text{such that}\quad
\Pi(u_1) = \Pi(u_2)
\quad\text{and}\quad
S(u_1) \neq S(u_2),
\]
then the statement crosses a projection boundary.

The computation remains formally valid,
but loses physical interpretability.

\subsection{Result}

Projection admissibility is a structural criterion.

It does not restrict formal mathematics.
It restricts physical interpretation.

This rule applies universally, independent of model, scale, or domain.
