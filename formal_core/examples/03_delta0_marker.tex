\section{Delta Zero as Interpretability Marker}

This application demonstrates the role of $\Delta_0$
as a formal marker separating admissible physical interpretation
from purely formal continuation.

\subsection{Setup}

Let
\[
\Pi : \mathcal{U} \rightarrow m_2
\]
be a projection into an observable calculation domain.

Consider a sequence of relational expressions
\[
S_n(u), \quad u \in \mathcal{U},
\]
that can be formally extended for increasing $n$.

\subsection{Loss of Decidability}

Assume that for some critical index $n_0$
the projection no longer allows a unique inverse:
\[
\Pi^{-1}(S_{n_0}) \;\; \text{is not uniquely defined}.
\]

The expression remains syntactically valid,
but its physical interpretation becomes undecidable.

\subsection{Definition of $\Delta_0$}

$\Delta_0$ marks exactly this transition:
\[
S_{n_0} \;\longrightarrow\; \Delta_0
\]

$\Delta_0$ is not a numerical value,
but a decision marker indicating interpretability loss.

\subsection{Distinction from Limits}

$\Delta_0$ is:

\begin{itemize}
\item not a limit in the analytic sense,
\item not a regularization scheme,
\item not a physical boundary or constant.
\end{itemize}

It does not replace a calculation.
It annotates its validity range.

\subsection{Relation to the Black Boundary}

$\Delta_0$ appears in the vicinity of a black boundary.

While the black boundary describes structural
loss of injectivity,
$\Delta_0$ marks the point where a concrete statement
can no longer be assigned physical meaning.

\subsection{Result}

Formal continuation remains allowed beyond $\Delta_0$.
Physical interpretation does not.

$\Delta_0$ enforces clarity without enforcing termination.
