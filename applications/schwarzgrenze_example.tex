\section{Schwarzgrenze Example}

Let
\[
\Pi : \mathcal{U} \rightarrow \mathcal{M}
\]
be a projection.

Define the Schwarzgrenze $\Sigma$ as the set
\[
\Sigma := \{ u \in \mathcal{U} \mid \Pi \text{ is not injective at } u \}.
\]

\subsection*{Structural Meaning}

For any
\[
u \in \Sigma
\]
there exist at least two distinct relational states
\[
u_1 \neq u_2
\]
such that
\[
\Pi(u_1) = \Pi(u_2).
\]

\subsection*{Observable Consequence}

All observables in $\mathcal{M}$ remain well-defined.

What is lost is the unique relational origin.

\subsection*{Interpretational Boundary}

The Schwarzgrenze does not invalidate calculations.

It invalidates uniqueness of physical interpretation.

\subsection*{Formal Status}

Beyond $\Sigma$:
\begin{itemize}
\item relational structures remain consistent,
\item formal continuation is allowed,
\item physical attribution is no longer admissible.
\end{itemize}

\subsection*{Role in MACHWERK}

The Schwarzgrenze marks the boundary between
physically referable statements
and purely formal extensions.
