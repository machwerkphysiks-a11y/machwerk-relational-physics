\section{Black Boundary Application}

This application demonstrates the role of the black boundary
as a structural limit of physical interpretability.

\subsection{Setup}

Let
\[
\Pi : \mathcal{U} \rightarrow m_2
\]
be a projection into an observable calculation space.

Assume two distinct relational configurations
\[
u_1 \neq u_2 \in \mathcal{U}
\]

\subsection{Projection Collapse}

Suppose the projection satisfies
\[
\Pi(u_1) = \Pi(u_2)
\]

The observable state in $m_2$ is identical,
although the underlying relational configurations differ.

\subsection{Black Boundary Identification}

The set of all such configurations defines the black boundary
\[
\Sigma := \{ u \in \mathcal{U} \mid \Pi \text{ is non-injective} \}
\]

\subsection{Consequence}

Within $\Sigma$:

\begin{itemize}
\item formal calculations remain valid
\item observable quantities remain well-defined
\item relational states are no longer uniquely reconstructible
\end{itemize}

\subsection{Interpretation Rule}

Any statement that distinguishes between $u_1$ and $u_2$
is physically inadmissible beyond this boundary.

The black boundary marks loss of interpretability,
not loss of consistency.
