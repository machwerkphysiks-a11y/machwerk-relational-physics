\section{Black Boundary Application}

This application demonstrates the role of the black boundary
as a structural limit of physical interpretability.

\subsection{Setup}

Let
\[
\Pi : \mathcal{U} \rightarrow m_2
\]
be a projection into an observable calculation space.

Assume two distinct relational configurations
\[
u_1 \neq u_2 \in \mathcal{U}.
\]

\subsection{Projection Collapse}

Suppose the projection satisfies
\[
\Pi(u_1) = \Pi(u_2).
\]

The observable state in $m_2$ is identical,
although the underlying relational configurations differ.

This implies that the observable description
cannot distinguish between $u_1$ and $u_2$.

\subsection{Black Boundary Identification}

The set of all such configurations defines the black boundary
\[
\Sigma := \{\, u \in \mathcal{U} \mid \Pi \text{ is non-injective} \,\}.
\]

At $\Sigma$, injectivity of the projection is lost.

\subsection{Consequences}

Within $\Sigma$:

\begin{itemize}
\item formal calculations remain valid,
\item observable quantities remain well-defined,
\item relational states are no longer uniquely reconstructible.
\end{itemize}

No contradiction arises.
What is lost is uniqueness of physical interpretation.

\subsection{Interpretation Rule}

Any physical statement $S$ that depends on the distinction
between $u_1$ and $u_2$ is inadmissible within $\Sigma$.

Formally, a statement $S$ is physically admissible if and only if
\[
\forall u_1, u_2 \in \mathcal{U} :
\Pi(u_1) = \Pi(u_2) \;\Rightarrow\; S(u_1) = S(u_2).
\]

If this condition is violated,
the statement exceeds the domain of physical interpretability.

\subsection{Operational Meaning}

The black boundary does not forbid calculation.
It forbids interpretation.

Beyond $\Sigma$, mathematical continuation is permitted,
but physical meaning must not be assigned.

\subsection{Role in Model Evaluation}

The black boundary provides a formal criterion to:

\begin{itemize}
\item reject over-interpreted results,
\item compare models at their limits of validity,
\item separate mathematical structure from physical content.
\end{itemize}

The black boundary is a diagnostic marker,
not a physical object and not a dynamical process.
