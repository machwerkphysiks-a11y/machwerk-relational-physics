\section{Definition of $m_2$}

\subsection{Role of $m_2$}

$m_2$ denotes the domain in which
projected relational expressions
remain uniquely back-referencable
to the relational space $\mathcal{U}$.

It is not a space,
not a dimension,
and not a physical layer.

$m_2$ is a \emph{validity domain}
for physical statements.

\subsection{Formal definition}

Let
\[
\Pi : \mathcal{U} \rightarrow m
\]
be a projection.

Then $m_2$ is defined as:
\[
m_2 := \{\, u \in \mathcal{U} \mid \Pi \text{ is injective at } u \,\}
\]

Equivalently:
\[
u \in m_2
\;\Longleftrightarrow\;
\exists!\, \Pi^{-1}(\Pi(u))
\]

Only within $m_2$
does a projected statement
possess a unique relational origin.

\subsection{Physical admissibility}

A physical statement is admissible
if and only if
all relational quantities involved
are evaluated within $m_2$.

Outside $m_2$,
formal computation remains possible,
but physical interpretation is undefined.

\subsection{Dynamic character}

$m_2$ is not fixed.

Its extent depends on
experimental resolution,
measurement access,
and validated relational reconstruction.

With increasing empirical reach,
$m_2$ may expand,
but it never exhausts $\mathcal{U}$.

\subsection{Relation to the Schwarzgrenze}

The boundary of $m_2$
is precisely the Schwarzgrenze $\Sigma$.

\[
\partial m_2 = \Sigma
\]

Crossing $\Sigma$
marks the transition
from physical validity
to purely formal continuation.
