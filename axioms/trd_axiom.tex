\section{TRD Axiom}

\textbf{TRD Axiom (Three-Relation Admissibility)}

A physically admissible comparative statement requires
a minimum of three distinct relational quantities.

Let
\[
R_1, R_2, R_3
\]
be three distinguishable process rates.

Only ratios of rates are admissible:
\[
X_{ij} := \frac{R_i}{R_j}
\]

A comparison is admissible if and only if
it can be expressed as a relation among at least three such ratios.

\subsection*{Formal Condition}

A relational statement $S$ is admissible if and only if
there exists a triple
\[
(R_1, R_2, R_3)
\]
such that
\[
S = S(X_{12}, X_{23}, X_{31})
\]

Statements depending on fewer than three relationally independent
rates are underdetermined and not invariant under reference change.

\subsection*{Invariance Property}

The TRD structure is invariant under global rescaling:
\[
\forall \lambda > 0:\quad
(R_1, R_2, R_3) \mapsto (\lambda R_1, \lambda R_2, \lambda R_3)
\]

All admissible relations remain unchanged under this transformation.

\subsection*{Exclusion}

TRD does not define:
\begin{itemize}
    \item time evolution,
    \item causality,
    \item dynamics,
    \item physical mechanisms.
\end{itemize}

It defines only the minimal structural condition
for relational comparability.

\subsection*{Boundary Robustness}

In regions where pairwise relations lose uniqueness
(e.g. non-injective projection regions),
TRD comparisons remain formally well-defined
as long as relational consistency is preserved.

\subsection*{Role}

The TRD axiom defines the minimal admissible comparison structure
within the formal framework.
It precedes and constrains all higher-order statements.
