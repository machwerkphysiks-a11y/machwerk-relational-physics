\section{CRA Axiom}

\textbf{CRA Axiom (Consistency of Relational Applicability)}

A relational statement is physically admissible if and only if
all involved quantities

\begin{enumerate}
    \item are defined within the same relational validity domain,
    \item are subject to the same projection boundary,
    \item are combined without implicit category transitions.
\end{enumerate}

Let
\[
\mathcal{R} = \{ r_1, r_2, \dots, r_n \}
\]
be a finite set of relational expressions.

A statement
\[
S(\mathcal{R})
\]
is said to be \emph{physically admissible} if and only if there exists
a validity domain $m_2$ such that
\[
\exists\, m_2 \;:\;
\forall r_i \in \mathcal{R}, \quad r_i \in m_2.
\]

If no such domain exists, the statement remains formally computable,
but loses physical interpretability.

Violation of the CRA axiom does not indicate a mathematical error.
It indicates a category error in the interpretation of the result.

The CRA axiom does not restrict computation.
It restricts attribution of physical meaning.
