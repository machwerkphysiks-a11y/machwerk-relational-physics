\section{Projection}

\subsection{Definition}

A \emph{projection} is a mapping from the relational full space $\mathcal{U}$
into a restricted computational domain.

Formally:
\[
\Pi : \mathcal{U} \rightarrow m
\]

where $m$ denotes a rechenraum in which quantities
are representable, comparable, and operable.

The projection $\Pi$ is not assumed to be injective, surjective,
or information-preserving.

\subsection{Interpretational role}

Physical interpretation does not arise in $\mathcal{U}$,
but exclusively through projection.

A relational configuration $u \in \mathcal{U}$
acquires physical meaning only insofar as
$\Pi(u)$ admits a consistent interpretation
within the target domain $m$.

Different elements of $\mathcal{U}$
may correspond to identical projected representations:
\[
u_1 \neq u_2 \quad \text{with} \quad \Pi(u_1) = \Pi(u_2)
\]

Such equivalence is a property of the projection,
not of the underlying relations.

\subsection{Information reduction}

Every admissible projection is informationally reducing.

This reduction is structural, not epistemic.
It does not reflect lack of knowledge,
but loss of distinguishability
imposed by the projection itself.

No projection can increase relational resolution.

\subsection{Context dependence}

The form and properties of $\Pi$
depend on the chosen observational context.

There is no privileged or universal projection.
Different physical descriptions correspond
to different projection choices.

Comparisons between projections are only meaningful
within domains where their images overlap.

\subsection{No inverse in general}

An inverse mapping
\[
\Pi^{-1}
\]
does not exist in general.

Inverse reconstruction is possible only
in regions where $\Pi$ is injective.
These regions are defined later
as domains of physical admissibility.

Outside such regions,
formal continuation remains possible,
but physical interpretation is suspended.
