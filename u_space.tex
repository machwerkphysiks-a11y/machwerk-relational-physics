\section{U-Space}

\subsection{Definition}

Let $\mathcal{U}$ denote the \emph{relational full space} (U-space).

$\mathcal{U}$ is defined as the set of all admissible relational configurations
between distinguishable processes, without presupposing space, time,
metric structure, geometry, scale, or localization.

Formally:
\[
\mathcal{U}
:= \left\{
\mathcal{R}
\;\middle|\;
\mathcal{R} \text{ is a finite or countable set of mutually consistent relations}
\right\}
\]

Elements of $\mathcal{U}$ are not objects, events, states, or points.
They are exclusively relational structures.

\subsection{Non-ontological status}

$\mathcal{U}$ carries no physical ontology.

It is not a physical space, not a hidden dimension,
and not a background structure of reality.
It serves solely as a formal reference domain
in which relational consistency is defined
prior to any notion of observability.

No statement about existence, causality, or dynamics
is permitted at the level of $\mathcal{U}$.

\subsection{Absence of intrinsic structure}

The U-space possesses:

\begin{itemize}
    \item no metric,
    \item no topology,
    \item no ordering parameter,
    \item no preferred coordinates,
    \item no intrinsic dimensionality.
\end{itemize}

Any such structures arise only through projection
into restricted rechenräume
and are not properties of $\mathcal{U}$ itself.

\subsection{Processes and relations}

Let $R_i$ denote distinguishable processes.
Processes have no absolute magnitude or scale.

Only relations between processes are admissible.
A minimal relational expression is given by:
\[
X_{ij} := \frac{R_i}{R_j}
\]

Absolute values of $R_i$ are undefined and physically meaningless.

All admissible quantities in the framework
are constructed from stable relational ratios.

\subsection{Role of $\mathcal{U}$ in the framework}

$\mathcal{U}$ functions as the maximal formal domain
from which all physically interpretable descriptions
must ultimately be projected.

The descriptive reach of physics does not depend on extending $\mathcal{U}$,
but on identifying which subsets of $\mathcal{U}$
admit stable and invertible projections.

$\mathcal{U}$ itself is invariant under all admissible transformations
defined later in the framework.

No computation performed solely in $\mathcal{U}$
carries physical meaning without additional constraints.
