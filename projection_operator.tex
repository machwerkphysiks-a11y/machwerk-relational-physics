\section{Projection Operator ($\Pi$)}

Physical observability is not fundamental.
It arises through projection.

A projection is a mapping
\[
\Pi : \mathcal{U} \rightarrow m
\]
from the relational full space $\mathcal{U}$ into an observable computation domain $m$.

The projection operator $\Pi$ is, in general:
\begin{itemize}
    \item information-reducing,
    \item context-dependent,
    \item not necessarily injective,
    \item not necessarily surjective.
\end{itemize}

Distinct relational configurations in $\mathcal{U}$ may produce identical
observable results after projection.

Therefore, equality in the observable domain does not imply identity
in the relational domain.

Projection introduces structure:
space, time, dimensionality, and measurable quantities
are properties of the projected domain,
not of $\mathcal{U}$.

No inverse projection exists globally.
An inverse mapping $\Pi^{-1}$ is only defined locally
in regions where $\Pi$ remains injective.

All physical interpretation is restricted to such regions.
